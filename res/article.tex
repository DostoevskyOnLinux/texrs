%%%%%%%%%%%%%%%%%%%%%%%%%%%%%%%%%%%%%%%%%%%%%%%%%%
%%% This template was drafted by Ethan Barry, but
%%% has been created with tidbits from many diff-
%%% erent sources. You may have it under the terms
%%% of the BSD 2-clause license. Go wild.
%%% -----
%%% It is also part of the program called
%%% 'texrs'. The source code of the program
%%% excepting this file is licensed under the
%%% terms of the GPLv3.
%%%%%%%%%%%%%%%%%%%%%%%%%%%%%%%%%%%%%%%%%%%%%%%%%%

%%%%%%% PRELUDE %%%%%%%%%%%%%%%%%%%%%%%%%%%%%%%%%%

%-----% FORMATTING %-----------------------------%
 \documentclass[letterpaper,titlepage,12pt]{article} % USA's usage.
%\documentclass[a4paper,10pt]{article} % European usage.
\usepackage[letterpaper,
            left=1in,
            right=1in,
            top=1in,
            bottom=1in,
            footskip=.25in]{geometry}

%-----% USEFUL TOOLS %---------------------------%
\usepackage{enumitem}
\usepackage[hidelinks]{hyperref}
\usepackage{tabularx}
\usepackage[utf8]{inputenc}
\usepackage[T1]{fontenc}
\usepackage{parskip}
\usepackage{fontspec}
\usepackage{setspace}
\usepackage{multicol}
\usepackage{caption}

%-----% RESOURCES & BIBLIOGRAPHY %---------------%
\usepackage[noibid, backend=biber, isbn=false, doi=false]{biblatex-chicago}
\usepackage{url}
\addbibresource{refs.bib}

%-----% IMAGES & GRAPHICS %----------------------%
\usepackage{graphicx}
\usepackage{tikz}
\graphicspath{{graphics/}}

\captionsetup{
  font=small,
  labelfont=bf,
  tableposition=top
}

%-----% SET SPACING OPTIONS %--------------------%
\addtolength{\topmargin}{-5pt}
\setlength{\parindent}{25pt}
\renewcommand{\baselinestretch}{1.5}
\frenchspacing

%-----% HEADER & FOOTER %------------------------%
\usepackage{fancyhdr}
\pagestyle{fancy}
\fancyhf{}
\setlength{\headheight}{15pt}
\renewcommand{\headrulewidth}{0pt}
\fancyhead[R]{<YOURNAME> \thepage}

%-----% FONT OPTIONS %---------------------------%
% |     Here you can select from a number of fonts
% | that most computers and LaTeX distributions
% | should provide. While I prefer EBGaramond &
% | Palatino, there are many others available in a
% | standard TeX distribution. See:
% | https://tug.org/FontCatalogue/ for many more
% | choices.
%-----% ----- %----------------------------------%
\defaultfontfeatures{Contextuals=Alternate}
%-----% SERIF %---------%
% \usepackage{palatino}
  \usepackage{ebgaramond}
% \usepackage{times}
% \usepackage{charter}

%-----% SANS-SERIF %----%
% \usepackage{helvet}
% \usepackage[sfdefault]{noto-sans}
% \usepackage[default]{sourcesanspro}

% \setromanfont[Mapping=tex-text]{Linux Libertine O}
% \setsansfont[Mapping=tex-text]{DejaVu Sans}
% \setmonofont[Mapping=tex-text]{DejaVu Sans Mono}

%%%%%%%%%%%%%%%%%%%%%%%%%%%%%%%%%%%%%%%%%%%%%%%%%%
%-----% THE DOCUMENT %---------------------------%
\begin{document}

%-----% TITLEPAGE %------------------------------%
% | https://en.wikibooks.org/wiki/LaTeX/Title_Creation
% | This titlepage allows for good manual control
% | over how it looks. Recompile it, add or sub-
% | tract spacing, and see how it looks.
%-----% ----- %----------------------------------%
\begin{titlepage}
	\centering
%	\includegraphics[width=0.15\textwidth]{example-image-1x1}\par\vspace{1cm}
	{\scshape\Large University of Foo \\ at Bar\par}
%	\vspace{1cm}
%	{\scshape\Large Final year project\par}
	\vspace{5cm}
	{\scshape\LARGE The Big Long Title,\\ \& Many More Important\\ Words\par}
	\vfill
	{\Large\itshape Fizz F. Fuzz\\}
%	\vspace{2mm}
	{\Large\scshape Art 9000: Advanced \LaTeX ~Typesetting\\}
%	\vspace{2mm}
	{\Large\scshape 25 Octember 2000\\}

\vspace{5cm}
\end{titlepage}

%-----% TABLE OF CONTENTS %----------------------%
\begin{titlepage}
    \tableofcontents
    \listoffigures
\end{titlepage}
%-----% ----- %----------------------------------%

%-----% THE ACTUAL DOCUMENT %--------------------%
% |     Aaand here's where you write your text.
% | Good luck!
%-----% ----- %----------------------------------%



%-----% FIGURE PAGES %---------------------------%
% | If you need a separate figures page, then un-
% | comment the following code.
%-----% ----- %----------------------------------%
% \newpage
% \center{\scshape Figures}
% \begin{figure}[h!]
%  \centering
%  \includegraphics[width=0.7\textwidth]{example.jpg}
%  \caption{Leonardo Da Example. {\it Coolest Painting Ever}. Oil on canvas. Callerie degli Uffizi}
%  \label{fig:cool_leonardo}
% \end{figure}

%-----% BIBLIOGRAPHY %---------------------------%
% |     Prints on a new page in keeping with the
% | Chicago Manual of Style's guidelines. You can
% | change this, too.
%-----% ----- %----------------------------------%
\newpage
\center{\scshape Bibliography}
\printbibliography[heading=none]
%-----% END OF THE DOCUMENT %--------------------%
% |     Well, that wasn't so bad! Congrats!
%%%%%%%%%%%%%%%%%%%%%%%%%%%%%%%%%%%%%%%%%%%%%%%%%%
\end{document}

