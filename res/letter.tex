%%%%%%%%%%%%%%%%%%%%%%%%%%%%%%%%%%%%%%%%%%%%%%%%%%
%%% This template was drafted by Ethan Barry, but
%%% has been created with tidbits from many diff-
%%% erent sources. You may have it under the terms
%%% of the BSD 2-clause license. Go wild.
%%% -----
%%% It is also part of the program called
%%% 'texrs'. The source code of the program
%%% excepting this file is licensed under the
%%% terms of the GPLv3.
%%%%%%%%%%%%%%%%%%%%%%%%%%%%%%%%%%%%%%%%%%%%%%%%%%

%%%%%%% PRELUDE %%%%%%%%%%%%%%%%%%%%%%%%%%%%%%%%%%

%-----% FORMATTING %-----------------------------%
 \documentclass[letterpaper,12pt]{article}
%\documentclass[a4paper,10pt]{article} % European usage.
\pagestyle{empty}
\usepackage[letterpaper,
            left=0.8in,
            right=0.8in,
            top=0.7in,
            bottom=0.7in]{geometry} % Be sure to change this in Europe.

%-----% USEFUL TOOLS %---------------------------%
\usepackage{lipsum}
\usepackage{setspace}
\usepackage{paracol}

%-----% IMAGES & GRAPHICS %----------------------%
\usepackage{tikz}
\usepackage{graphicx}
\usepackage{xcolor}
\usepackage[skins,breakable]{tcolorbox}
\graphicspath{{graphics/}}

%-----% COLOR DEFINITIONS %----------------------%
\definecolor{longline}{RGB}{0,40,89}

%-----% FONT OPTIONS %---------------------------%
% |     Here you can select from a number of fonts
% | that most computers and LaTeX distributions
% | should provide. While I prefer EBGaramond &
% | Palatino, there are many others available in a
% | standard TeX distribution. See:
% | https://tug.org/FontCatalogue/ for many more
% | choices.
%-----% ----- %----------------------------------%
% \defaultfontfeatures{Contextuals=Alternate} % Enable only with XeLaTeX.
%-----% SERIF %---------%
% \usepackage{palatino}
  \usepackage{ebgaramond}
% \usepackage{times}
% \usepackage{charter}

%-----% SANS-SERIF %----%
% \usepackage{helvet}
% \usepackage[sfdefault]{noto-sans}
% \usepackage[default]{sourcesanspro}

% \setromanfont[Mapping=tex-text]{Linux Libertine O}
% \setsansfont[Mapping=tex-text]{DejaVu Sans}
% \setmonofont[Mapping=tex-text]{DejaVu Sans Mono}

%%%%%%%%%%%%%%%%%%%%%%%%%%%%%%%%%%%%%%%%%%%%%%%%%%
%-----% THE DOCUMENT %---------------------------%
\begin{document}

%-----% TITLEBAR %-------------------------------%
\begin{flushleft}
    {\Large\scshape UNIVERSITY OF FOO AT BAR }\\
    \vspace*{0.06in}
    {\scshape DEPARTMENT OF AWESOMENESS }
\end{flushleft}

\setlength{\columnsep}{0.5in}
\setcolumnwidth{4.5in, 1in}

%-----% THE ADDRESS COLUMN %---------------------%
% |     This column section holds the address, &
% | an image if you choose to include one.
%-----% ----- %----------------------------------%
\begin{paracol}{2}
    \vspace*{0.05in} % Balance height ratios.
    \onehalfspacing
    \noindent Dr. Ludovicus Humperdink \\
    3300 University Drive \\
    Happytown, YZ, 01010 \\
    Some other text here?
    \switchcolumn
    \vspace*{-1in}\hspace*{-0.8in}\includegraphics[width=2in]{example_logo.png}
    % Don't forget to replace the above with your graphic.    ^^^^^^^^^^^^^^^^
    % You can use the spacing options on the image to change its position. Use
    % the [width=<number>in] option to change its size on the page. Alternatively
    % comment that line out to drop the graphic.
\end{paracol}
%-----% A REALLY LONG LINE %---------------------%
% | This draws a line from one border to the other.
% | Adjust the color in the color definitions block.
% | Adjust the thickness here; it's the number at the end.
%-----% ----- %----------------------------------%
\vspace*{-0.5in}
\noindent\textcolor{longline}{\makebox[\linewidth]{\rule{\textwidth}{0.8pt}}}
\vspace*{0.5in}

\begin{paracol}{2}
%-----% THE ACTUAL DOCUMENT %--------------------%
% |     Aaand here's where you write your text.
% | Remove the `\lipsum[] blocks, and replace them
% | with your own paragraphs. They're just here so
% | you have some sample text to work with while
% | you adjust the image placement.
%-----% ----- %----------------------------------%
    \noindent\textbf{Your title here}
    \\ \\
    \lipsum[1]
    \\ \\
    \lipsum[2]
    \\ \\
    Regards,
    \\ \\ \\
    {\scshape Fuzz B. Buzz}\\
    Position / Title

%-----% IMAGES & GRAPHICS %----------------------%
% | Here's a nice little column for your info, &c.
%-----% ----- %----------------------------------%
    \switchcolumn
    {\footnotesize\vspace{0.03in}
    \noindent 25 Octember, 2000
    \\
    1010 Memory Lane \\
    Vimville, WQ, 11011
    \\ \\
    fuzz@fizz.com \\
    www.foobar.com
    \\ \\
    }
\end{paracol}
%-----% END OF THE DOCUMENT %--------------------%
% |     Well, that wasn't so bad! Good job!
%-----% ----- %----------------------------------%
%%%%%%%%%%%%%%%%%%%%%%%%%%%%%%%%%%%%%%%%%%%%%%%%%%
\end{document}
