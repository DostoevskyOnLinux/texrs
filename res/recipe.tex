\documentclass[letterpaper,twoside,10pt,landscape]{article}
% Ordered Alphabetically
\usepackage[english]{babel}
\usepackage{ebgaramond}
\usepackage{eso-pic}
\usepackage{enumitem}
\usepackage{fancyhdr}
\usepackage[top=0.5in,
            marginparwidth=48pt,
            paperwidth=4in,
            paperheight=6in]{geometry}
\usepackage{layout}
\usepackage{marginnote}
\usepackage{multicol}
\usepackage{pgfornament}
\usepackage{calligra}
\usepackage{textcomp}
\usepackage{titlesec}
\usepackage{xcolor}

\pdfgentounicode=1
\pagenumbering{gobble}
\fancyhf{}
\setlength{\headheight}{0pt}
\setlength{\footskip}{0pt}
\setlength{\textheight}{240pt}

\setlength{\multicolsep}{10pt}
\setlength{\marginparsep}{15pt}
\setlength{\marginparpush}{15pt}

%%%%%%%%%%%%%%%%%%%%%%%%%%%%%%%%%%%%%%%%%%%%%%%%%%
%-----% FORMATTING MACROS %----------------------%

% Section format.
\titleformat{\section}{
  \vspace{-4pt}\scshape\raggedright\large
}{}{0em}{}[\titlerule\vspace{-5pt}]

% Recipe items.
\newcommand{\RItem}[1]{
  \item\small{{#1}}
}

\newcommand{\RItemListStart}{\begin{itemize}}
\newcommand{\RItemListEnd}{\end{itemize}\vspace{-5pt}}

\newcommand{\RecipeInstructionsStart}{
\begin{enumerate}[label={\textit{\arabic{enumi}:}},
                  leftmargin = 2em]}
\newcommand{\RecipeInstructionsEnd}{\end{enumerate}}
%%%%%%%%%%%%%%%%%%%%%%%%%%%%%%%%%%%%%%%%%%%%%%%%%%

%%%%%%%%%%%%%%%%%%%%%%%%%%%%%%%%%%%%%%%%%%%%%%%%%%
%-----% THE DOCUMENT %---------------------------%
\begin{document}

\begin{multicols}{2}
% Title here:
\noindent \pgfornament[width=0.25in]{12}

\columnbreak

{\calligra{\fontsize{16pt}{2}\selectfont Homemade Sandwich Bread}}
\end{multicols}

\small{\textbf{Servings:} 2 loaves}
% More 'recipe metadata' can go here!

\section{Ingredients}
\begin{multicols}{2}
 \RItemListStart
  \renewcommand\labelitemi{---}
  \RItem{6 cups all-purpose flour}
  \RItem{2 tbsp. instant or active dry yeast}
  \RItem{2 tbsp. granulated sugar}
  \RItem{1 tbsp. kosher salt (or $1\frac{1}{2}$ tsp. table salt)}
  \RItem{2 cups very warm water}
  \RItem{$\frac{1}{4}$ cup extra-virgin olive oil, divided}
  \item[\vspace{\fill}]
 \RItemListEnd
\end{multicols}
\section{Instructions}
 \RecipeInstructionsStart
  \RItem{In a large mixing bowl, whisk together the flour, yeast, sugar, \& salt.
         Then add in the water \& 2 tbsp. of the oil; mix with a wooden spoon until
         a ``shaggy'' dough forms.\marginnote{\textit{No idea what shaggy is doing describing food...}}}
  \RItem{Turn out onto a clean surface; knead until smooth, or for about 4 minutes.}
  \RItem{Form a smooth dough ball and return to the mixing bowl. Cover with a clean towel
  \& let rise in a warm place for about 15 minutes.}
  \RItem{Divide the dough in half \& form into two loaf shapes, then place them
  in standard-size loaf pans.}
  \RItem{Drizzle a tbsp. of the remaining olive oil over each, then slash the tops
  of the loaves about $\frac{1}{4}$ in. deep with a sharp knife.}
  \RItem{Place the loaves in the middle rack of a \textbf{cold} oven with a pan
  of hot tap water below it.}
  \RItem{Close the oven; set to 400\textdegree, \& start a timer for 40 min.}
  \RItem{After this, transfer the loaves from the pans to a cooling rack}
  \item[\vspace{\fill}]
 \RecipeInstructionsEnd

\end{document}
